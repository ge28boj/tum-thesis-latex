\section{RISC-V ISA}

RISC-V, which stands for the fifth generation of reduced instruction set computer, was first developed by Krste Asanović in 2010 at UC Berkeley.
Unlike ARM-based microprocessors, RISC-V is an open-source ISA. Hence, it has been considered "the Linux of microprocessors".
As a matter of fact, RISC-V is a lot more than that. The groundbreaking innovation lies in its modular design.

\noindent For a long time, incremental ISA, including x86 and ARM, has been the industry convention. In other words, architects keep adding instructions but never removing for the sake of backward compatibility.  
This leads to tremendous complexity in chip design. Besides, hardware costs increase since more silicon areas are required for decode and execution of a wider variety of instructions. And most importantly,
this monolithic design approach introduces less flexibility. There is the analogy - you pay for the whole buffet, but all you want is salad.   

\section{ETISS}
\section{ELF File Format}
\section{Machine Learning Deployment}
\section{Valgrind}